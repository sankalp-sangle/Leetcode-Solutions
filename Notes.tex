\documentclass[11pt]{book}    

\usepackage{hyperref}
\hypersetup{
    colorlinks=true,
    linkcolor=blue,
    filecolor=magenta,      
    urlcolor=blue,
}

\parindent0pt  \parskip10pt             % make block paragraphs
\raggedright                            % do not right justify

\title{\bf LeetCode Solutions}    % Supply information
\author{Sankalp Sangle}              %   for the title page.
\date{Last updated on \today}                           %   Use current date. 

% Note that book class by default is formatted to be printed back-to-back.
\begin{document}                        % End of preamble, start of text.
\frontmatter                            % only in book class (roman page #s)
\maketitle                              % Print title page.
\tableofcontents                        % Print table of contents
\mainmatter                             % only in book class (arabic page #s)

\part{LeetCode Top Interview Questions}                   % Print a "part" heading

\chapter{Easy}                % Print a "chapter" heading
Link: \href{https://leetcode.com/explore/interview/card/top-interview-questions-easy/}{LeetCode Top Interview Questions: Easy section}. 

\section{Arrays}                  % Print a "section" heading
Link: \href{https://leetcode.com/explore/interview/card/top-interview-questions-easy/92/array/}{Arrays}

\subsection{26. Remove Duplicates from Sorted Array}
\href{https://leetcode.com/problems/remove-duplicates-from-sorted-array/}{Link to question},
\href{https://leetcode.com/submissions/detail/332869256/}{Link to submission}
\paragraph{Concepts}
Two pointer
\paragraph{Algorithm description}
\begin{itemize}
    \item Maintain a read pointer and a write pointer, both starting from zero.
    \item Advance the write pointer until you see a new value or reach end of array.
    \item Write value at write location into read location.
    \item Return read.
\end{itemize}

\subsection{122. Best Time to Buy and Sell Stock II}
\href{https://leetcode.com/problems/best-time-to-buy-and-sell-stock-ii/}{Link to question},
\href{https://leetcode.com/submissions/detail/332894775/}{Link to submission}
\paragraph{Concepts} Greedy

\paragraph{Algorithm description}
\begin{itemize}
    \item Construct a consecutive elements difference array
    \item Return sum of all positive elements in difference array
\end{itemize}

\subsection{189. Rotate Array}
\href{https://leetcode.com/problems/rotate-array/}{Link to question},
\href{https://leetcode.com/submissions/detail/333064713/}{Link to submission approach 1},
\href{https://leetcode.com/submissions/detail/333059899/}{Link to submission approach 2}
\paragraph{Concepts}
Cyclic replacements, Implementation
\paragraph{Approach 1 description}
\begin{itemize}
    \item Maintain a visited array and a pointer initialized to 0
    \item while pointer + k is not visited, replace arr[pointer + k] with arr[pointer]. Update
    pointer to pointer + k. Set pointer + k to visited, increment a numberOfChanges variable.
    \item Increment pointer by 1
    \item Keep doing this while numberOfChanges less than size of array.
\end{itemize}
\paragraph{Approach 2 description}
\begin{itemize}
    \item Reverse the entire array
    \item Reverse from start to start + k
    \item Reverse from start + k to end
\end{itemize}

\subsection{217. Contains Duplicate}
\href{https://leetcode.com/problems/contains-duplicate/}{Link to question},
\href{https://leetcode.com/submissions/detail/333069819/}{Link to submission}
\paragraph{Concepts}
Hash Table, Set
\paragraph{Algorithm description}
\begin{itemize}
    \item Initialize a Set
    \item For an element in array, if element in Set, return true
    \item else add element to Set
    \item If out of loop, return False
\end{itemize}
 
\subsection{136. Single Number}
\href{https://leetcode.com/problems/single-number/}{Link to question},
\href{https://leetcode.com/submissions/detail/333115980/}{Link to submission}
\paragraph{Concepts}
Bit Manipulation, XOR
\paragraph{Algorithm description}
\begin{itemize}
    \item Initialize an answer variable to 0
    \item For every element, XOR it to answer. Elements appearing twice get XOR'd out to zero
    \item Return answer
\end{itemize}

\subsection{350. Intersection of Two Arrays II}
\href{https://leetcode.com/problems/intersection-of-two-arrays-ii/}{Link to question},
\href{https://leetcode.com/submissions/detail/333119654/}{Link to submission approach 1},
\href{https://leetcode.com/submissions/detail/333123474/}{Link to submission approach 2}
\paragraph{Concepts}
Hash Table, Two Pointers
\paragraph{Approach 1 description}
\begin{itemize}
    \item Form an element:frequency mapping using map for smaller array (to save space)
    \item Traverse bigger array
    \item If frequency of element less than 0, add to answer. Decrement frequency
\end{itemize}
\paragraph{Approach 2 description}
\begin{itemize}
    \item If arrays are sorted, use two pointers p1 and p2
    \item If nums1[p1] == nums2[p2], add to answer and increment both
    \item Else if nums1[p1] is smaller, increment p1. Else increment p2
    \item Keep doing until reach end of either array
\end{itemize}

\subsection{66. Plus One}
\href{https://leetcode.com/problems/plus-one/}{Link to question},
\href{https://leetcode.com/submissions/detail/333134799/}{Link to submission}
\paragraph{Concepts}
Array
\paragraph{Algorithm description}
\begin{itemize}
    \item Initialize a carry variable to 1
    \item Traverse array from the end. \\ digit[i] = carry + digit mod 10, carry = carry + digit div 10
    \item Finally, if carry is not zero, insert carry at start of array
\end{itemize}

\subsection{283. Move Zeroes}
\href{https://leetcode.com/problems/move-zeroes/}{Link to question},
\href{https://leetcode.com/submissions/detail/333138944/}{Link to submission}
\paragraph{Concepts}
Two Pointers
\paragraph{Algorithm description}
\begin{itemize}
    \item Maintain a read and a write pointer, both initialized to 0
    \item if read end has zero, increment read end
    \item else, copy read end to write end and increment both
    \item After read end reaches end, set all numbers from write end to end as 0
\end{itemize}

\subsection{1. Two Sum}
\href{https://leetcode.com/problems/two-sum/}{Link to question},
\href{https://leetcode.com/submissions/detail/333142473/}{Link to submission approach 1},
\href{https://leetcode.com/submissions/detail/333144602/}{Link to submission approach 2}
\paragraph{Concepts}
Hash Table, Two Pointer
\paragraph{Approach 1 description}
\begin{itemize}
    \item Create an element:indices mapping
    \item Sort the array
    \item Use two pointers to search for a particular sum
    \item Once you find the sum, pop index from left pointer, and pop index from right pointer
    \item Return indices
\end{itemize}
\paragraph{Approach 2 description}
\begin{itemize}
    \item Create a hashmap of int, int
    \item Iterate the array with i as looping variable
    \item If element in hashmap, return (hashmap[element], i)
    \item Else insert hashmap[target - element] = i
\end{itemize}

\subsection{36. Valid Sudoku}
\href{https://leetcode.com/problems/valid-sudoku/}{Link to question},
\href{https://leetcode.com/submissions/detail/333176948/}{Link to submission}
\paragraph{Concepts}
Hash Table, Set
\paragraph{Algorithm description}
\begin{itemize}
    \item Create sets to hold numbers for each row, col and square.
    \item Traverse the sudoku
    \item If a number is already in the row, col, square, return False
    \item Else, come out of loop and return true
\end{itemize}

\subsection{48. Rotate Image}
\href{https://leetcode.com/problems/rotate-image/}{Link to question},
\href{https://leetcode.com/submissions/detail/333188499/}{Link to submission}
\paragraph{Concepts}
Array, Circular Permutation
\paragraph{Algorithm description}
\begin{itemize}
    \item Do a counterclockwise circular permutation as mentioned in solution
    \item Pure implementation problem. No algorithmic skill.
\end{itemize}

\newpage

\section{Strings}
Link: \href{https://leetcode.com/explore/interview/card/top-interview-questions-easy/127/strings/}{Strings}

\subsection{344. Reverse String}
\href{https://leetcode.com/problems/reverse-string/}{Link to question},
\href{https://leetcode.com/submissions/detail/333295707/}{Link to submission}
\paragraph{Concepts}
Two Pointers
\paragraph{Algorithm description}
\begin{itemize}
    \item Set a left pointer to start of string, right pointer to end
    \item Swap left and right. Increment left, decrement right
    \item Do while l less than r
\end{itemize}

\subsection{7. Reverse Integer}
\href{https://leetcode.com/problems/reverse-integer/}{Link to question},
\href{https://leetcode.com/submissions/detail/333300400/}{Link to submission}
\paragraph{Concepts}
Two Pointers
\paragraph{Algorithm description}
\begin{itemize}
    \item Reverse the integer by converting to a string
    \item Store result in long
    \item If stored result is outside integer limits, return 0
    \item Else return the reversed number
\end{itemize}

\subsection{387. First Unique Character in a String}
\href{https://leetcode.com/problems/first-unique-character-in-a-string/}{Link to question},
\href{https://leetcode.com/submissions/detail/333317093/}{Link to submission}
\paragraph{Concepts}
Hash Map
\paragraph{Algorithm description}
\begin{itemize}
    \item Construct element frequency mapping
    \item Traverse the string from the start, if frequency of a char is 1, return index
    \item If reach end of string, return -1
\end{itemize}

\subsection{242. Valid Anagram}
\href{https://leetcode.com/problems/valid-anagram/}{Link to question},
\href{https://leetcode.com/submissions/detail/333323752/}{Link to submission}
\paragraph{Concepts}
Hash Map, Counting Sort
\paragraph{Algorithm description}
\begin{itemize}
    \item Traverse through s1, incrementing frequency counts
    \item Traverse through s2, decrementing frequency counts
    \item If all counts are zero, return true. Else false.
\end{itemize}

\subsection{125. Valid Palindrome}
\href{https://leetcode.com/problems/valid-palindrome/}{Link to question},
\href{https://leetcode.com/submissions/detail/333333971/}{Link to submission}
\paragraph{Concepts}
Two Pointers
\paragraph{Algorithm description}
\begin{itemize}
    \item Maintain a left and a right pointer
    \item Before comparing the two, ensure left and right both are pointing to an
    alphanumeric character
\end{itemize}

\subsection{28. Implement strStr()}
\href{https://leetcode.com/problems/implement-strstr/}{Link to question},
\href{https://leetcode.com/submissions/detail/333634575/}{Link to Approach 1},
\href{https://leetcode.com/submissions/detail/333729270/}{Link to Approach 2}
\paragraph{Concepts}
Two Pointers, Rabin-Karp Algorithm, Rolling Hash
\paragraph{Approach 1 description}
\begin{itemize}
    \item Traverse haystack until you find a character matching with first character of
    needle
    \item Once match is found, keep checking for further characters 
    until either there's a mismatch or you reach end of arrays
    \item Return index accordingly
\end{itemize}
\paragraph{Approach 2 description - Rabin-Karp}
\begin{itemize}
    \item Hash the needle using a hash function that is easy to be "rolled", that is it is
    easy to compute hash for next window if hash for previous window is known
    \item Traverse the haystack using window of length needle.length(). Hash the window and 
    compare with needle hash. If matched, return the index of start of window
    \item See implementation carefully, very interesting. Also see LeetCode solution article.
\end{itemize}

\subsection{38. Count and Say}
\href{https://leetcode.com/problems/count-and-say/}{Link to question},
\href{https://leetcode.com/submissions/detail/333834219/}{Link to submission}
\paragraph{Concepts}
Recursion, Two Pointers
\paragraph{Algorithm description}
\begin{itemize}
    \item Base case: n = 1, return "1"
    \item Get the answer for n-1
    \item Traverse through answer of n-1
    \item For each consecutive list of same elements, add the count, followed by the element
    \item Return answer
\end{itemize}

\subsection{14. Longest Common Prefix}
\href{https://leetcode.com/problems/longest-common-prefix/}{Link to question},
\href{https://leetcode.com/submissions/detail/333844730/}{Link to submission}
\paragraph{Concepts}
Implementation
\paragraph{Algorithm description}
\begin{itemize}
    \item Initialize answer string to ""
    \item Find length of smallest string
    \item For i from 0 to min length - 1
    \item Traverse through all the characters at ith positions
    \item If different, return answer
    \item If same, add character to answer
\end{itemize}

\newpage

\section{Linked Lists}
Link: \href{https://leetcode.com/explore/interview/card/top-interview-questions-easy/93/linked-list/}{Linked Lists}

\subsection{237. Delete Node in a Linked List}
\href{https://leetcode.com/problems/delete-node-in-a-linked-list/}{Link to question},
\href{https://leetcode.com/submissions/detail/333854173/}{Link to submission}
\paragraph{Concepts}
Trick
\paragraph{Algorithm description}
\begin{itemize}
    \item Copy value of next node into current node
    \item Set next ptr of current node to next ptr of next node
\end{itemize}

\subsection{19. Remove Nth Node From End of List}
\href{https://leetcode.com/problems/remove-nth-node-from-end-of-list/}{Link to question},
\href{https://leetcode.com/submissions/detail/333863765/}{Link to submission}
\paragraph{Concepts}
Two Pointer
\paragraph{Algorithm description}
\begin{itemize}
    \item To do it in one pass, let a forward pointer advance n steps
    \item Then, start forwarding a slow pointer as well as the forward pointer one at a time
          until forward reaches the end
    \item delete the slow pointer node
\end{itemize}

\subsection{206. Reverse Linked List}
\href{https://leetcode.com/problems/reverse-linked-list/}{Link to question},
\href{https://leetcode.com/submissions/detail/333870150/}{Link to iterative approach},
\href{https://leetcode.com/submissions/detail/333909164/}{Link to recursive approach}
\paragraph{Concepts}
Implementation
\paragraph{Approach 1 description}
\begin{itemize}
    \item Initialize a prev = NULL, and a curr = head
    \item While head is not NULL, do a cyclic swap between curr.next, prev, and curr.
    \item Return prev
\end{itemize}
\paragraph{Approach 2 description}
\begin{itemize}
    \item If head is NULL or head.next is NULL return head
    \item l = reversed list for head.next
    \item head.next.next = head, head.next = NULL. Return l
\end{itemize}

\subsection{21. Merge Two Sorted Lists}
\href{https://leetcode.com/problems/merge-two-sorted-lists/}{Link to question},
\href{https://leetcode.com/submissions/detail/334181938/}{Link to iterative submission},
\href{https://leetcode.com/submissions/detail/334165928/}{Link to recursive submission}
\paragraph{Concepts}
Two Pointers
\paragraph{Algorithm description Iterative}
\begin{itemize}
    \item Make a dummy node, and let tmp = dummynode
    \item Keep appending the smaller of the two lists to the dummy node and advance the pointers
        accordingly
    \item If one of the lists becomes NULL, append the other list to dummy node
    \item Return next of tmp
\end{itemize}
\paragraph{Algorithm description Recursive}
\begin{itemize}
    \item If either of lists is NULL, return the other
    \item if l1 is smaller, get answer to (l1.next, l2) and set it as l1.next. Return l1
    \item Else get answer to (l1, l2.next) and set it as l2.next. Return l2
\end{itemize}

\subsection{234. Palindrome Linked List}
\href{https://leetcode.com/problems/palindrome-linked-list/}{Link to question},
\href{https://leetcode.com/submissions/detail/334192848/}{Link to submission}
\paragraph{Concepts}
Reverse a linked list, Two Pointers
\paragraph{Algorithm description}
\begin{itemize}
    \item Reverse the second half of the linked list
    \item Compare nodewise the head of linked list and the head of reversed list
        to check for palindrome
\end{itemize}

\subsection{141. Linked List Cycle}
\href{https://leetcode.com/problems/linked-list-cycle/}{Link to question},
\href{https://leetcode.com/submissions/detail/334195990/}{Link to submission}
\paragraph{Concepts}
Hare and Tortoise, Two Pointers
\paragraph{Algorithm description}
\begin{itemize}
    \item Initialize a slow and a fast pointer
    \item Advance slow by 1, fast by 2
    \item If slow and fast meet, there's a cycle. Else if fast reaches end, there's no cycle.
\end{itemize}

\newpage

\section{Trees}
Link: \href{https://leetcode.com/explore/interview/card/top-interview-questions-easy/94/trees/}{Trees}

\subsection{104. Maximum Depth of Binary Tree}
\href{https://leetcode.com/problems/maximum-depth-of-binary-tree/}{Link to question},
\href{https://leetcode.com/submissions/detail/334227706/}{Link to recursive submission},
\href{https://leetcode.com/submissions/detail/334260483/}{Link to iterative submission}
\paragraph{Concepts}
Recursion, Stack
\paragraph{Algorithm description Recursive}
\begin{itemize}
    \item If root is null, return 0
    \item Else return 1 + max(maxDepth(left), maxDepth(right))
\end{itemize}
\paragraph{Algorithm description Iterative}
\begin{itemize}
    \item If root is null, return 0
    \item Initialize stack holding pair of TreeNode and depth
    \item Push \{root, 1\}
    \item While stack is not empty, get top of stack
    \item If top is leaf, compare with maxDepth
    \item Push children if any with depth = 1 + parent depth
\end{itemize}

\subsection{98. Validate Binary Search Tree}
\href{https://leetcode.com/problems/validate-binary-search-tree/}{Link to question},
\href{https://leetcode.com/submissions/detail/334300642/}{Link to iterative submission},
\href{https://leetcode.com/submissions/detail/334297136/}{Link to recursive submission}
\paragraph{Concepts}
Top-Down
\paragraph{Algorithm description (for recursive/iterative)}
\begin{itemize}
    \item Approach is a top-down one
    \item At every node, check if node.val is between a range of [small, large]
    \item If not, return False
    \item else check left subtree for range[small, node.val] and check right subtree for range[node.val, large]
    \item Return the AND of the above two
\end{itemize}

\subsection{101. Symmetric Tree}
\href{https://leetcode.com/problems/symmetric-tree/}{Link to question},
\href{https://leetcode.com/submissions/detail/335199504/}{Link to recursive submission},
\href{https://leetcode.com/submissions/detail/335199304/}{Link to iterative submission}
\paragraph{Concepts}
Top-Down
\paragraph{Algorithm description (for recursive/iterative)}
\begin{itemize}
    \item Top down approach
    \item Check if leftTree.val == rightTree.val
    \item If true, check for leftTree.left, rightTree.right and leftTree.right, rightTree.left
    \item Else, return False
\end{itemize}

\subsection{102. Binary Tree Level Order Traversal}
\href{https://leetcode.com/problems/binary-tree-level-order-traversal/}{Link to question},
\href{https://leetcode.com/submissions/detail/335213120/}{Link to submission}
\paragraph{Concepts}
Top-Down, BFS
\paragraph{Algorithm description}
\begin{itemize}
    \item Push root into a queue
    \item At beginning of an iteration, take size of queue
    \item Pop out \#size items from queue, while adding their children to queue
    \item Add to level
    \item Add level to final answer
\end{itemize}

\subsection{108. Convert Sorted Array to Binary Search Tree}
\href{https://leetcode.com/problems/convert-sorted-array-to-binary-search-tree/}{Link to question},
\href{https://leetcode.com/submissions/detail/335241263/}{Link to submission}
\paragraph{Concepts}
Recursion, Preorder
\paragraph{Algorithm description}
\begin{itemize}
    \item call procedure with left = 0, right = arr.size() - 1
    \item if left > right, return NULL
    \item construct node for middle element
    \item node.left = procedure(left, middle-1), node.right = procedure(middle+1, right)
    \item return node
\end{itemize}

\newpage

\section{Sorting and Searching}
Link: \href{https://leetcode.com/explore/featured/card/top-interview-questions-easy/96/sorting-and-searching/}{Sorting and Searching}

\subsection{88. Merge Sorted Array}
\href{https://leetcode.com/problems/merge-sorted-array/}{Link to question},
\href{https://leetcode.com/submissions/detail/335252401/}{Link to submission}
\paragraph{Concepts}
Two Pointers
\paragraph{Algorithm description}
\begin{itemize}
    \item Create a copy array for nums1
    \item Maintain write pointer for nums1, p1 for nums1copy, p2 for nums2
    \item Write smaller of p1, p2 into nums1. Advance smaller and write head.
    \item Once out of the loop, see which array still has elements remaining. Add them to nums1
\end{itemize}

\subsection{278. First Bad Version}
\href{https://leetcode.com/problems/first-bad-version/}{Link to question},
\href{https://leetcode.com/submissions/detail/335254058/}{Link to submission}
\paragraph{Concepts}
Binary Search
\paragraph{Algorithm description}
\begin{itemize}
    \item set left as 0, right as n - 1
    \item while l <= r
    \item if mid is bad, right = middle - 1
    \item else left = middle + 1
    \item Once you come out of loop, return l
\end{itemize}

\newpage

\section{Dynamic Programming}
Link: \href{https://leetcode.com/explore/featured/card/top-interview-questions-easy/97/dynamic-programming/569/}{Dynamic Programming}

\subsection{70. Climbing Stairs}
\href{https://leetcode.com/problems/climbing-stairs/}{Link to question},
\href{https://leetcode.com/submissions/detail/335398986/}{Link to submission}
\paragraph{Concepts}
Dynamic Programming
\paragraph{Algorithm description}
\begin{itemize}
    \item Ways to reach ith step = ways to reach i-1 th step plus ways to reach i-2 th step
\end{itemize}

\subsection{121. Best Time to Buy and Sell Stock}
\href{https://leetcode.com/problems/best-time-to-buy-and-sell-stock/}{Link to question},
\href{https://leetcode.com/submissions/detail/335403058/}{Link to submission}
\paragraph{Concepts}
Dynamic Programming
\paragraph{Algorithm description}
\begin{itemize}
    \item Maintain a smallest stock price seen yet variable
    \item Update maxProfit = max(maxProfit, current price - maxProfit)
\end{itemize}

\subsection{53. Maximum Subarray}
\href{https://leetcode.com/problems/maximum-subarray/}{Link to question},
\href{https://leetcode.com/submissions/detail/335407441/}{Link to submission}
\paragraph{Concepts}
Dynamic Programming
\paragraph{Algorithm description}
\begin{itemize}
    \item Maintain a current sum variable, denoting the highest sum possible that contains the element at the index
    \item Maintain a highest sum variable, denoting the highest sum encountered among the current sums
\end{itemize}

\subsection{198. House Robber}
\href{https://leetcode.com/problems/house-robber/}{Link to question},
\href{https://leetcode.com/submissions/detail/335418653/}{Link to submission},
\href{https://leetcode.com/submissions/detail/335415699/}{Link to submission (space optimized)}
\paragraph{Concepts}
Dynamic Programming
\paragraph{Algorithm description}
\begin{itemize}
    \item Maintain a dp array with dp[0] = nums[0], dp[1] = max(nums[0], nums[1]).
    dp[i] denotes maximum amount that can be robbed with first i+1 houses
    \item dp[i] = max(dp[i-1], dp[i-2] + nums[i])
    \item Finally return dp[n-1]
\end{itemize}

\newpage

\section{Design}
Link: \href{https://leetcode.com/explore/featured/card/top-interview-questions-easy/98/design/}{Design}

% \subsection{}
% \href{}{Link to question},
% \href{}{Link to submission}
% \paragraph{Concepts}

% \paragraph{Algorithm description}
% \begin{itemize}
%     \item 
% \end{itemize}
\end{document}                          % The required last line
